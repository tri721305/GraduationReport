\par{\textit{Trong chương này, tôi đưa ra những dẫn chứng về nhu cầu du lịch đặc biệt của giới trẻ trong đời sống hiện nay. Từ đó nêu ra lí do lựa chọn đề tài, tôi trình bày mục tiêu của đề tài và cuối cùng là ý nghĩa của đề tài.}}

\section{Giới thiệu đề tài và lý do chọn đề tài}
Du lịch được ghi nhận như là 1 sở thích, một hoạt động nghỉ ngơi mang đến sự tích cực của con người. Ngày nay, du lịch trở thành một nhu cầu không thể thiếu trong cuộc sống văn hoá, xã hội. Về mặt kinh tế, du lịch đã trỏ thành một trong những ngành kinh tế “hái tiền” lợi hại nhất của các nước. Mạng lứoi du lịch được thiết lập tất cả các quốc gia trên thế giới. \\
Ông bà ta còn có câu: \\
"	Đi 1 ngày đàng học 1 sàng khôn  "\\
Du lịch còn mang lại nhiều thông tin bổ ích, hiểu biết được nhiều kiến thức.Vì vậy để đưa du lịch đến gần tay hơn với mọi người, web du lịch đã nghĩ ra trên ý tưởng:\\
\par{\textit{"Thiết lập kế hoạch du lịch 1 cách chi tiết cho bản thân, nhóm, công ty…. Cho cách mục đi du lịch đâu? Ở đó làm gì? Ăn gi? uống gì? Địa điểm nào đẹp? Đi bao nhiêu ngày? Thời gian thực hiện các hoạt động trong ngày?  Tất cả đều nămf trong mục đăng kí kế hoạch đi đu lịch của bản thân, nhóm, công ty. Web sẽ hiện thực hoá suy nghĩ kế hoạch suông trong suy nghĩ thành những dữ liệu cụ thể( có đưa ra gợi ý). Ngoài ra bạn có thể chia sẻ plan mình cho những người muốn “rủ rê” đi cùng."}}

\section{Mục tiêu đề tài}
Với nhu cầu du lịch cao của mọi người hiện nay, đặc biệt là giới trẻ cùng với việc đã tham gia trải nghiệm  vào nhiều chuyến du lịch thực tế mà không có kế hoạch trước làm rất tốn thời gian, không chủ động trong sinh hoạt, du lịch .
\par{Website " Du lịch Goz " hướng đến việc khắc phục về việc đó. Người sử dụng có thể chủ động lên kế hoạch trước cho cá nhân hoặc nhóm: Đi đâu? Ăn gì? Trải nghiêm ở đâu? Di chuyển bằng phương tiện gì? Trong khung giờ như thế nào? }

\par{Mỗi bài thảo luận các thành viên bình luận, đóng góp ý kiến qua comment}
\par{Có thể đặt trước dịch vụ: thuê xe, thuê khách sạn, nhà hàng. Đánh giá dịch vụ đã sử dụng bằng điểm * và comment.}
\par{Sau khi đi du lịch sẽ có bài review về chuyến đi mà mình đã từng đi, dịch vụ từng ở, đánh giá nhận xét của bản thân về chuyến đi. Mỗi bài review đều có điểm tương tác thông qua cách chấm điểm trên thang 1-5 và bình luận}






% \section{Giới thiệu về đề tài và đặt vấn đề}
%     \par
%     Phòng Công tác chính trị Sinh viên có chức năng tổ chức thực hiện công tác giáo dục chính trị, tư tưởng cho cán bộ, viên chức và sinh viên toàn trường, đảm bảo đúng đường lối chính sách của Đảng, pháp luật của nhà nước; góp phần đào tạo sinh viên trở thành con người toàn diện có đạo đức, tri thức, sức khoẻ, thẩm mỹ và nghề nghiệp, trung thành với lý tưởng độc lập dân tộc và chủ nghĩa xã hội, dưới sự lãnh đạo của Hiệu trưởng và Đảng uỷ trường. Phòng Công tác Chính trị Sinh viên thực hiện các nhiệm vụ sau:
%     \begin{itemize}
%         \item Giáo dục chính trị - tư tưởng, đạo đức, lối sống và tổ chức các hoạt động rèn luyện cho sinh viên: triển khai các đợt sinh hoạt chính trị, học tập Nghị quyết, chủ trương của Đảng và chính sách pháp luật của nhà nước cho cán bộ, viên chức và sinh viên; nắm bắt tình hình diễn biến tư tưởng của sinh viên và đề xuất với Đảng uỷ, Hiệu trưởng có biện pháp tuyên truyền, giáo dục phù hợp và kịp thời.
%         \item Tổ chức thực hiện công tác đánh giá điểm rèn luyện theo năm học. 
%         \item Phối hợp với Đoàn Thanh niên, Hội Sinh viên, Ban liên lạc Cựu sinh viên tổ chức các hoạt động rèn luyện thân thể, xã hội, văn hoá, văn nghệ, thể dục thể thao.
%         \item Tổ chức tuần sinh hoạt công dân cho sinh viên vào đầu năm học.
%         \item Tổ chức thực hiện chế độ chính sách cho sinh viên: trợ cấp xã hội, miễn giảm học phí.
%         \item Phối hợp với các đơn vị, Đoàn Thanh niên, Hội sinh viên theo dõi và chịu trách nhiệm đề xuất công tác thi đua – khen thưởng, kỷ luật sinh viên với Hiệu trưởng.
%         \item Kiểm tra việc chấp hành quy chế học sinh - sinh viên nội trú, ngoại trú; kiến nghị xử lý các trường hợp vi phạm.
%     \end{itemize}
%     \par
%     Trong đề tài luận văn này sẽ tập trung giải quyết những vấn đề: Phối hợp với Đoàn Thanh niên, Hội Sinh viên tổ chức các hoạt động rèn luyện thân thể, xã hội, văn hoá, văn nghệ, thể dục thể thao. Xây dựng sàn hoạt động ngoại khóa sinh viên. Tổ chức thực hiện công tác đánh giá điểm rèn luyện, ngày CTXH theo năm học.
%     \par
%     \textbf{Kết luận chung: } Sinh viên khoa KH\&KT Máy tính nói riêng cũng như sinh viên trường Đại học Bách Khoa nói chung thực sự cần một hệ thống giải quyết được vấn đề đã được đề ra ở trên. Việc xây dựng hệ thống Phòng Công tác Chính trị Sinh viên là thực sự cần thiết. 

%     \section{Khó khăn, thử thách}
%     \begin{itemize}
%         \item Độ ổn định của hệ thống.
%         \item Tính bảo mật của hệ thống.
%         \item Lượng sinh viên sử dụng lớn.
%         \item Tính tiện lợi, đầy đủ các tính năng tuy nhiên vẫn dễ dàng trong sử dụng, vận hành.
%         \item Một hệ thống có thể áp dụng cho nhiều khoa với đặc điểm, cách hoạt động riêng.
%         \item Hỗ trợ được trên nhiều nền tảng khác nhau.
%     \end{itemize}
%     \section{Mục tiêu luận văn}
%     Mục tiêu của đề tài này là xây dựng một hệ thống (website, ứng dụng) Du lịch có đầy đủ các tính năng như sau:
%     % \begin{itemize}
%     %     \item Kênh thông tin cho người dùng có thể xem các bài viết, hoạt động, địa điểm du lịch đang nổi bật hiện nay.
%     %     \item Tính năng tạo kế hoạch, sự kiện trên hệ thống: người dùng có thể tạo các hoạt động một cách chi tiết trong ngày kèm theo một bài viết review đi kèm, các chức năng hỗ trợ ban truyền thông tạo, soạn thảo bài viết. Ngoài ra, hệ thống sẽ thực hiện được:
%     %     \begin{itemize}
%     %         \item Khi ban truyền thông, ban tổ chức đăng hoặc tổ chức sự kiện, hoạt động. Mỗi hoạt động, bài đăng sẽ phù hợp với những sinh viên khác nhau, hệ thống sẽ có nhiệm vụ gửi thông báo đến các sinh viên có nhu cầu cũng như gợi ý hoạt động cho sinh viên trên ứng dụng hoặc thông qua các kênh khác nhau.
%     %         \item Nhắc nhở sinh viên tham gia hoạt động khi hoạt động sắp diễn ra.
%     %         \item Cập nhật tình hình hoạt động của sinh viên cũng như là nhắc nhở sinh viên tham gia hoạt động.
%     %     \end{itemize}
%     %     \item Trợ lý sinh viên mỗi Khoa sẽ thực hiện được việc sắp xếp GVCN và thành phần ban cán sự lớp cho mỗi lớp trong mỗi năm học.
%     %     \item Khởi tạo bộ tiêu chí rèn luyện dành riêng cho mỗi Khoa dựa trên bộ tiêu chí chuẩn của Trường đưa ra. Mỗi khoa tạo, xoá, sửa, cập nhật các tiêu chí xét duyệt.
%     %     \item Tính năng cập nhật ngày CTXH và ĐRL sau mỗi sự kiện, hoạt động kết thúc.
%     %     \item Thực hiện công tác chấm điểm rèn luyện mỗi năm học online trên hệ thống. Sau đó được ban cán sự lớp và GVCN duyệt trước khi gửi kết quả về mỗi Khoa tổng hợp trước khi gửi về Trường.
%     % \end{itemize}

%     % Từ những mục tiêu tổng quát trên, nhóm nghiên cứu phân loại thành những mục tiêu cụ thể ứng với từng loại đối tượng người dùng trên một đơn vị, cụ thể như sau:
%     % \begin{itemize}
%     % \item 
%     % Các đối tượng quản lý:
%     % \begin{itemize}
%     %     \item 
%     %     Quản lý hệ thống.
%     %     \item 
%     %     Quản lý sự kiện, thêm, xoá, sửa, duyệt sự kiện.
%     %     \item
%     %     Quản lý các danh mục sự kiện để phân loại.
%     %     \item 
%     %     Cấu hình các sự kiện, thay đổi email để thông báo cho các sinh viên đăng ký tham gia sự kiện, tham gia sự kiện và cảnh báo sinh viên.
%     %     \item 
%     %     Tạo bản nháp sự kiện quá trình duyệt được trải qia 3 bước duyệt sự kiện bởi người có quyền, duyệt điểm rèn luyện, duyệt ngày CTXH có phù hợp với sự kiện đó hay không.
%     %     \item 
%     %     Cấu hình điểm rèn luyện, email nhắc nhở sinh viên phản hồi điểm rèn luyện nộp phiếu chấm điểm rèn luyện.
%     %     \item 
%     %     Tạo mới bộ tiêu chí cho mỗi Khoa, có thể copy từ Khoa khác hoăc từ bộ tiêu chí chuẩn của Trường. Thay đổi thêm, xoá sửa từng tiêu chí.
% %         \item 
% %         Vai trò ban cán sự lớp có nhiệm vụ xét duyệt lại các phiếu chấm điểm rèn luyện của các thành viên trong lớp của mình. Xem các phiếu chấm có đúng hay sai, có quyền chỉnh sửa điểm và để lại phản hồi.
% %         \item 
% %         Tương tự giáo viên cũng có thể duyệt điểm rèn luyện cho sinh viên của lớp chủ nhiệm. Ngoài ra còn có quyền phân cônn thành phần ban cán sự lớp. 
% %         \item 
% %         Trợ lý sinh viên mỗi Khoa sẽ có quyền gán GVCN và thành phần ban cán sự lớp cho mỗi lớp vào mỗi năm học cho Khoa. Tiếp nhận và xử lý phản hồi điểm rèn luyện của sinh viên.
% %         \item 
% %         Xuất ra danh sách điểm rèn luyện của Lớp, của Khoa.
        
% %     \end{itemize}
% %      \item 
% %     Đối tượng là sinh viên:
% %     \begin{itemize}
% %         \item 
% %         Xem các thông tin về điểm rèn luyện mỗi năm học.
% %         \item
% %         Thực hiện việc chấm điểm rèn luyện mỗi năm học.
% %     \end{itemize}
% % \end{itemize}
%     \section{Phạm vi đề tài}
%     Phạm vi các đối tượng mà đề tài hướng đến bao gồm:
%     \begin{itemize}
%         \item Người dùng ở mọi lứa tuổi, đặc biệt là giới trẻ từ 20 đến 30 tuổi.
%         \item Cộng tác viên khách sạn, homestay, quán ăn, .....
%     \end{itemize}