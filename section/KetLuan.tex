\section{Kết quả đạt được}
\subsection{Nghiệp vụ}
Thông qua việc phân tích các yêu cầu của người dùng, nghiên cứu, ứng dụng các công nghệ hiện tại, đồng thời tham gia vào các chuyến du lịch Nhóm đã thống kê yêu cầu nghiệp vụ của ứng dụng và  xây dụng được một hệ thống thông tin Website du lịch hỗ trợ người dùng lên chuyến đi chi tiết, kết bạn và chia sẽ chuyến đi cho bạn bè người thân cùng tham gia. 

\subsection{Công nghệ}
Trong quá trình thực hiện đề tài, nhóm đã có cơ hội được tìm hiểu ngôn ngữ Javascript, framework ReactJS và NodeJS để tạo nên website du lịch hỗ trợ người dùng tạo kế hoạch chi tiết cho chuyến đi . Bên cạnh đó, nhómcòn sử dụng một số thư viện khác như Boostrap, date-picker, ....

\subsubsection{Website}
\begin{itemize}
    \item Website có giao diện thân thiện, trực quan, dễ sử dụng, tương thích với hầu hết các trình duyệt web.
    \item Hiển thị thông tin cho người dùng một cách đầy đủ và chi tiết về từng kế hoạch, hoạt động, ... giúp ngừơi dùng có thể chọn nơi du lịch đáp ứng tốt nhu cầu bản thân.
    \item Hỗ trợ khách hàng xem lại lịch sử các plan đã tạo và tham gia.
    \item Hỗ trợ người dùng, giải đáp thắc mắc người dùng qua email
    \item Gửi thông báo sau khi đăng ký về điện thoại và email cho khách hàng.
    \item Hỗ trợ người dùng có thể tương tác kết bạn với nhau.
    \item Hỗ trợ người dùng có thể tham gia chuyến đi của người khác một cách dễ dàng
%     \item Quản lý thông tin học viên và thẻ tập chính xác, tránh trường hợp thiếu sót xảy ra.
%     \item Quản lý tiền lương của nhân viên tự động mỗi tháng với thông tin về buổi dạy đầy đủ và chi tiết từng lớp, từng buổi (bao gồm buổi dạy chính và cả buổi dạy thay).
%     \item Thống kê doanh thu của trung tâm qua từng tháng.
%     \item Thống kê số lượng học viên đăng ký học khóa học nào nhiều nhất để đưa ra những chiến lược sau này.
%     \item Tự động gửi về email, số điện thoại thông báo nhắc học viên thanh toán thẻ tập và một số thông báo khác khi người dùng đăng ký thẻ tập hoặc đăng ký là học viên mới,...
\end{itemize}


% \begin{itemize}
%     \item Xây dựng được trang tin tức tập trung cho người dùng ngoài hệ thống. Quản trị viên có thể tùy chỉnh các thành phần, các trang mới. Số lượng các thành phần có thể lựa chọn lớn. 
%     \item Xây dựng được hệ thống quản lý sự kiện. Bao gồm các chức năng cơ bản như tạo sự kiện, thêm sinh viên, điểm danh sự kiện cùng lúc tại nhiều thiết bị. Ngoài ra, sinh viên còn có thể theo dõi được các sự kiện mới.
%     \item Xây dựng được hệ thống quản lý điểm rèn luyện tích hợp với với sự kiện. Sinh viên có thể xem, theo dõi điểm rèn luyện trong kì, đăng ký tham gia hoạt động có điểm rèn luyện trên trang tin tức.
%     Quản trị viên có thể quản lý được điểm rèn luyện, thống kê, xuất tệp thông tin của sự kiện, sinh viên.
%     \item Xây dựng được hệ thống gợi ý cho sinh viên các sự kiện sắp diễn ra và tự động gửi mail cho sinh viên khi hệ thống phân tích được sinh viên phù hợp với sự kiện.
% \end{itemize}
% \section{Ưu điểm}
% Với các chức năng đã hiện thực được, hệ thống có một số ưu điểm sau:
% \begin{itemize}
%     \item Hệ thống đã được triển khai lên máy chủ, có thể được sử dụng bởi Khoa Khoa học và Kỹ thuật Máy tính.
%     \item Giao diện phần mềm đẹp và thân thiện
%     \item Đối tượng sử dụng phần mềm phong phú từ sinh viên, giáo viên cho đến các cán bộ nhà Trường.
%     \item Mã nguồn ở phía người dùng được tối ưu hóa.
% \end{itemize}
\section{Hạn chế}
Bên cạnh những chức năng đã hiện thực và ưu điểm, hệ thống vẫn tồn tại một số khuyết điểm sau.
\begin{itemize}
    \item Quá trình kiểm thử thực hiện bài bản song do hệ thống phức tạp và rất nhiều chức năng cho nên không thể tránh khỏi lỗi.
    \item Chưa tập trung vào vấn đề hiệu suất khi hiện thực nên khi hoạt động trong thời gian dài trên trình duyệt đôi khi xảy ra vấn đề thời gian đáp ứng không đủ nhanh.
    \item Chưa áp dụng các kỹ thuật bảo mật mạnh vào hệ thống.
\end{itemize}

\section{Hướng phát triển}
\begin{itemize}

    \item Hoàn thiện giao diện đẹp hơn nữa, thân thiện hơn nữa với người dùng.
    \item Tích hợp hệ thống đặt phòng khách sạn trực tuyến.
    \item Áp dụng công nghệ bảo mật mạnh vào hệ thống.
     \item Xây dựng ứng dụng di động. Với sự phát triển của điện thoại thông minh ngày nay, việc người dùng nói chung dành nhiều thời gian vào điện thoại
    hơn là máy tính. Việc xây dựng một ứng dụng di động là đi đúng với thực tế. Việc xây dựng ứng dụng di động có thể mở rộng khả năng áp dụng hệ thống vào
    thực tế. Các công việc, đơn cử là sự kiện, điểm danh sự kiện có thể thực hiện một cách dễ dàng. Cộng với việc hệ thống xây dựng trên ReactJS framework, 
    có thể dễ dàng ứng dụng React Native để xây dựng ứng dụng di động.
%     \item Nhân rộng mô hình hệ thống thông tin tới nhiều Khoa khác. Ngoài ra việc tối ưu hệ thống, giảm thiểu những dư thừa trong hiện thực, tính toán lại các tác vụ người dùng là một hướng phát triển tốt.
% \end{itemize}